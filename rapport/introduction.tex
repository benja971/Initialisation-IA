\section{Introduction}

    L'intelligence artificielle (IA) est un processus d'imitation de l'intelligence humaine qui repose sur la création et l'application d'algorithmes exécutés dans un environnement informatique dynamique. Son but est de permettre à des ordinateurs de penser et d'agir comme des êtres humains.

\vspace{0.5cm}

Pour y parvenir, trois composants sont nécessaires :

\begin{itemize}
    \item Des systèmes informatiques
    \item Des données avec des systèmes de gestion
    \item Des algorithmes d'IA avancés (code)
\end{itemize}

\vspace{0.5cm}

Pour se rapprocher le plus possible du comportement humain, l'intelligence artificielle a besoin d'une quantité de données et d'une capacité de traitement élevées.

Depuis au moins le premier siècle avant notre ère, l'Homme s'est penché sur la création de machines capables d'imiter le raisonnement humain. Le terme « intelligence artificielle » a été créé plus récemment, en 1955 par John McCarthy. En 1956, John McCarthy et ses collaborateurs ont organisé une conférence intitulée « Dartmouth Summer Research Project on Artificial Intelligence » qui a donné naissance au machine learning, au deep learning, aux analyses prédictives et, depuis peu, aux analyses prescriptives. Un nouveau domaine d'étude est également apparu : la science des données.
